\documentclass{ximera}

\title{Adding Fractions - Same Denominator}
\author{Mrs. Borrego}

\begin{document}
\begin{abstract}
    Now that we know what a fraction is, we can work on operations with fractions.
\end{abstract}
\maketitle

\begin{example} We can add fractions that have the same denominator like this: \\
Add $\frac{1}{4}+\frac{2}{4}$
\begin{explanation}
\begin{align*}
    \frac{1}{4}+\frac{2}{4}&=\frac{1+2}{4}\\
    &= \frac{3}{4}
\end{align*}
\end{explanation}
\end{example}


\begin{problem}
Add $\frac{1}{3}+\frac{1}{3}$.
\begin{multipleChoice}
\choice[correct]{$\frac{2}{3}$}
\choice{$\frac{2}{6}$}
\end{multipleChoice}
\end{problem}

\begin{problem} What is $\frac{1}{6}+\frac{2}{6}$?
\begin{selectAll}
\choice[correct]{$\frac{3}{6}$}
\choice[correct]{$\frac{1}{2}$}
\choice{$\frac{3}{12}$}
\end{selectAll}
\end{problem}

\begin{question}What is the name of the bottom number in a fraction?
\begin{multipleChoice}
\choice{bottomator}
\choice[correct]{denominator}
\end{multipleChoice}
\end{question}




\end{document}
